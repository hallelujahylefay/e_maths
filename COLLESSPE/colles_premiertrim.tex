\documentclass{article}
\usepackage[utf8]{inputenc}
\usepackage{amsmath}
\usepackage{array}
\usepackage{amssymb}
\newcommand{\mc}{\multicolumn{1}{c}}

\renewcommand{\arraystretch}{1.5}
\usepackage{nicematrix}
\usepackage{amsthm}
\usepackage{indentfirst}
\usepackage{stmaryrd}

\usepackage{mathtools}
\DeclarePairedDelimiter{\floor}{\lfloor}{\rfloor}
\DeclarePairedDelimiter{\ceil}{\lceil}{\rceil}
\DeclareMathOperator{\Tr}{Tr}
\DeclareMathOperator{\diag}{Diag}
\DeclareMathOperator{\adh}{Adh}
\DeclareMathOperator{\inte}{Int}
\begin{document}
\title{Colles du premier trimestre}
\author{YLF}
\date{Oct et plus, 2019}
\maketitle

\section{Semaine I : Algèbre générale}
\subsection*{Enoncés}
\subsubsection*{Exercice 1 (Sander).}
Démontrer que $(\mathbb{U}(\mathbb{Z}/p\mathbb{Z}), \times)$ est cyclique.
\subsection*{Solutions}
\subsubsection*{Exercice 1.}
On sait que $\mathbb{U}(\mathbb{Z}/p\mathbb{Z}) = \mathbb{Z}/p\mathbb{Z}\backslash\{0\}$, il faut donc montrer l'existence d'un élément d'ordre $(p-1)$. Ecrivons $p-1 = p_1^{\alpha_1}\ldots p_n^{\alpha_n}$, la décomposition en facteurs premiers de $p-1$. Il faut montrer qu'il existe un élément d'ordre $p_i^{\alpha_i}$ et cela pour tout $i\in\llbracket 1;n \rrbracket$.

...

Le produit de ces éléments est alors d'ordre $n-1$. Le résultat exposé est en vérité plus général, pour tout corps $K$, son groupe multiplicatif est cyclique et la démonstration est similaire.
\section{Semaine III : Analyse générale}
\subsection*{Enoncés}
\subsubsection*{Exercice 1 (Franklin).}
Posons
\begin{align*}
	\Delta : \left\{
		\begin{array}{@{}l@{\thinspace}l}
			\mathcal{F}(\mathbb{R}) \to \mathcal{F}(\mathbb{R})\\
			f \mapsto x \mapsto f(x+1)-f(x)
		\end{array}
		\right.
	\end{align*}
	\begin{enumerate}
		\item Montrer que pour tout $p\in \mathbb{N}$ et pour toute fonction $f \in \mathcal{C}^p(\mathbb{R})$, $x\in\mathbb{R}$, il existe $y \in [x;x+p]$  tel que $\Delta^p(f)(y) = f(x)$.
		\item Soit $\alpha \geq 0$ tel que pour tout $n\in\mathbb{N}$, $n^{\alpha} \in\mathbb{N}$. Que dire de $\alpha$ ?
	\end{enumerate}
\section{Semaine IV : Espaces vectoriels normés}
\subsection*{Exercice 1 (Houkari).}
Posons $E = \{f \in \mathcal{C}^{3}([0, 2], \mathbb{R}) : f(0) = f(1) = f(2)\}$ et $N : f \mapsto ||f^{(3)}||_{\infty, [0,2]}$. \begin{enumerate}
	\item Montrer que $N$ est une norme sur $E$.
	\item L'application $\displaystyle f\mapsto \int_{0}^2 f(t)\mathrm{d}t$ est-t-elle continue ?
\end{enumerate}
\subsection*{Solutions}
\subsubsection*{Exercice 1}
La première question ne pose aucune difficulté. L'application étudiée est linéaire, pour montrer qu'elle est continue, il suffit donc de montrer qu'elle est lipschitzienne en $0$. Autrement dit, peut-t-on trouver une constante $K>0$ telle que 
\begin{align*}
	\forall f \in E, \,\,\, \bigg|\int_{0}^2 f^{(3)}(t)\mathrm{d}t\bigg|\leq K ||f^{(3)}||_{\infty, [0,2]}
\end{align*}

Soit $f \in E$, l'intégration de l'inégalité de Taylor Lagrange assure que, 
\begin{align*}
	\bigg|\int_{0}^2 f(t)\mathrm{d}t\bigg|\leq \int_{0}^2 |f(t)|\mathrm{d}t\leq \int_{0}^2 (t|f'(0)| + \frac{t^2}{2}|f''(0)|+\frac{t^3}{6}|f^{(3)}(0)|)\mathrm{d}t
\end{align*}

Le théorème de Rolle appliqué deux fois assure l'existence de deux réels dans $[0, 2]$, $x_0, x_1$ tels que $f'(x_0) = 0$ et $f''(x_1) = 0$. Ainsi, 
\begin{align*}
	f''(0) = \int_{x_1}^{0}f^{(3)}(t)\mathrm{d}t\\
	f'(0) = \int_{x_0}^{0} \int_{x_1}^{x}f^{(3)}(t)\mathrm{d}t \mathrm{d}x
\end{align*}

Ainsi, on peut écrire directement que $|f'(0)| \leq 4||f^{(3)}||_{\infty}$ et $|f''(0)|\leq 2||f^{(3)}||_{\infty}$. Après majoration, on trouve que $K = (8+8/6\times 2+16/24)$ convient.

\section{Semaine V : Espaces vectoriels normés}
\subsection*{Enoncés}
\subsubsection*{Exercice 1 (Mme Santoni).}
Soit $B \in\mathbb{M}_p(\mathbb{C})$ telle que $(B^k)_k$ est bornée. Montrer que pour tout $n\in\mathbb{N}$, $A_n = \displaystyle \frac{1}{n+1}\sum_{k=0}^n B^k$ admet une valeur d'adhérence, notée $A$ et que $A^2 = A$.
\subsection*{Solutions}
\subsubsection*{Exercice 1.}
Notons $M$ un majoration de la suite $(B^k)_k$ pour une certaine norme (dim. finie), alors $(A_n)$ est majorée par $M$. Par le théorème de Bolzano-Weierstrass, $A_n$ admet une sous-suite, disons $(A_{\varphi(n)})$, convergeante vers $A$.

$BA = A$, en effet
\begin{align*}
	B A_{\varphi(n)} = A_{\varphi(n)} + \frac{1}{\varphi(n)+1}(B^{\varphi(n)+1} - B) \to A
	\end{align*}

	Puis $B^{k}A = A$, pour tout $k\in\mathbb{N}$ par récurrence immédiate. Enfin, en revenant à l'expression de $A_n$ et en la multipliant par $A$,
	\begin{align*}
		A_nA =A
	\end{align*}

	Par passage à la limite, le résultat est obtenu.

	\section{Semaine VII : Séries et Algèbre linéaire}
	\subsection*{Enoncés}
	\subsubsection*{Exercice 1 (Antoine Le Calvez).}
	Nature de $\sum \frac{e^{i\sqrt{n}}}{n}$ ?
	\subsubsection*{Exercice 2 (Antoine Le Calvez).}
	Soit $E$ un $\mathbb{K}$-ev de dimension $n$, trouver une famille libre de projecteurs de $E$ libre de cardinal maximum.
	\subsubsection*{Exercice 3 (Antoine Le Calvez).}
	Equivalent de $\sum (-1)^{n-1} \floor{\frac{n}{k}}$.
	\subsection*{Solutions}
	\subsubsection*{Exercice 1.}
Une méthode, posons pour tout $x \in \mathbb{R}^{+}$, $f(x) = \frac{e^{i\sqrt{x}}}{x}$ et calculons $||f'||_{\infty, \llbracket n; n+1\rrbracket}$ pour tout $n\in \mathbb{N}^*$. On obtient que $f'(x) = i\frac{e^{i\sqrt{x}}}{x^{3/2}} = O(x^{-3/2})$. Ainsi, $\sum ||f'||_{\infty}$ converge par Riemann. Utilisons donc la comparaison intégrale sur le reste 
\begin{align*}
	\sum \frac{e^{i\sqrt{n}}}{n} \sim \int \frac{e^{i\sqrt{x}}}{x}\mathrm{d}x = 2\int \frac{e^{i u}}{u}\mathrm{d}u 
\end{align*}

Aussi, $\frac{\mathrm{d}}{\mathrm{d}u} \frac{e^{iu}}{u} = \frac{e^{iu}}{u}$ et $\sum \frac{e^{i u}}{u}$ converge par le critère d'Abel. Ainsi, la série converge.

\subsubsection*{Exercice 3.}
	On obtient facilement $n\ln 2$.
	\section{Semaine VIII : Algèbre linéaire}
	\subsection*{Enoncés}
	\subsubsection*{Exercice 1 (Sander).}
	Déterminer l'ensemble des matrices $A\in \mathcal{M}_n(\mathbb{K})$ telles que pour tout $B\in\mathcal{M}_n(K)$, $\det A + B = \det A + \det B$.
	\subsection*{Exercice 2 (Sander).}
	Montrer que 
	\begin{align*}
		\forall \varphi \in \mathcal{M}_n(\mathbb{K})^*, \exists !A\in \mathcal{M}_n(\mathbb{K}) : \varphi = (X\mapsto \Tr(AX))
	\end{align*}
	Puis montrer que tout hyperplan de $\mathcal{M}_n(\mathbb{K})$ coupe $\mathcal{GL}_n(\mathbb{K})$.
	\subsubsection*{Exercice 3 (Sander).}
	Montrer que pour tout $A,\, B \in\mathcal{M}_n(\mathbb{K})$, $\det A^2 + B^2$ est positif.
	\subsection*{Solutions}
	\subsubsection*{Exercices 1, 2, 3}
	En posant $B = A$, on constate qu'ou bien $n=1$ ou bien $\det A = 0$. Si $\det A = 0$, alors on peut noter $r$ son rang, et alors $A = (C_1|\ldots|C_r|C^*_{r+1}|\ldots C^*_n)$ où disons pour simplifier sans que cela change quoi que ce soit au problème, que les $r$ premières colonnes sont libres. Complétons-les en une base de $\mathcal{M}_n(\mathbb{K})$, disons $(C_1, \ldots, C_n)$, alors posons $B = (0|\ldots|0|C_{r+1}|\ldots|C_n)$. Alors $\det A + B \neq 0$ et $\det B = 0$ si $r\geq 1$, si $r=0$ alors $A = 0$ et la propriété est vraie. Finalement, seule $A=0$ convient.

	Exercice classique, on pose l'endormorphisme $\psi : A \mapsto (X\mapsto \Tr(AX))$. On prouve que celle-ci est injective en utilisant les $E_{ij}$, en effet, $Tr(AE_{i,j}) = a_{j,i}$. Ainsi, $\psi$ est un isomorphisme, ce qui répond au problème. On utilise la caractérisation précédente et on se ramène aux matrices $J_r$. Par une analyse, on obtient que $B = \diag(1, \ldots, 1, -r+1, 1, \ldots 1)\in\mathcal{GL}_n(K)$ est telle que $\Tr(J_rB) = 0$. 

	On voit que $\det A^2 + B^2 = \det A-iB \det A+iB$, l'un est le conjugué de l'autre, c'est donc un nombre positif.
	\section{Semaine X : Réduction}
	\subsection*{Enoncés}
	\subsubsection*{Exercice 1 (Monier).}
	Soit $\mathcal{D}_n(\mathbb{C})$ l'ensemble des matrices diagonalisables dans $\mathbb{C}$ et $\mathcal{D}_n^+(\mathbb{C})$ le sous ensemble de $\mathcal{D}_n(\mathbb{C})$ des matrices qui admettent $n$-valeurs propres distinctes. Déterminer l'adhérence et l'intérieur de $\mathcal{D}_n(\mathbb{C})$ et $\mathcal{D}_n^+(\mathbb{C})$.
	\subsection*{Solutions}
	\subsubsection*{Exerice 1}
        
	Soit $\mathcal{D}_n(\mathbb{C})$ l'ensemble des matrices diagonalisables dans $\mathbb{C}$ et $\mathcal{D}_n^+(\mathbb{C})$ le sous ensemble de $\mathcal{D}_n(\mathbb{C})$ des matrices qui admettent $n$-valeurs propres distinctes. Déterminer l'adhérence et l'intérieur de $\mathcal{D}_n(\mathbb{C})$ et $\mathcal{D}_n^+(\mathbb{C})$.
	\subsection*{Solutions}
        
	\subsubsection*{Exerice 1}
	Montrons $\adh \mathcal{D}_n^+(\mathbb{C}) = \adh \mathcal{D}_n(\mathbb{C}) = \mathcal{M}_n(\mathbb{C})$. Soit $A\in\mathcal{M}_n(\mathbb{C})$ alors $A$ est trigonalisable, d'où 
	\begin{align*}
		A \sim \begin{pmatrix}   
  \lambda_1 & \star & \ldots & \star\\
	    & \lambda_2 & \ddots&\vdots\\
	    &&\ddots & \star \\
	    & & & \lambda_n
		\end{pmatrix} = A'
	\end{align*}
	et pour tout $k\in \mathbb{N}^*$, on définit 
	\begin{align*}
		A_k = \begin{pmatrix}
			\lambda_1-\frac{1}{k} & \star & \ldots & \star\\
					      & \lambda_2-\frac{2}{k} & \ddots & \vdots\\
					      &&\ddots & \star \\
					      & & & \lambda_n-\frac{n}{k}
		\end{pmatrix}
	\end{align*}

	Alors $A_k\to A'$ et APCR, $A_k \in\mathcal{D}_n^+(\mathbb{C})$ car les éléments de sa diagonale finissent par tous être distincts, d'où $\adh \mathcal{D}_n^+(\mathbb{C})=\mathcal{M}_n(\mathbb{C})$. Comme $\mathcal{D}_n^+(\mathbb{C})\subset \mathcal{D}_n(\mathbb{C})$, on a bien montré que $\adh \mathcal{D}_n^+(\mathbb{C}) = \adh \mathcal{D}_n(\mathbb{C}) = \mathcal{M}_n(\mathbb{C})$. 

	Montrons que $\mathcal{D}_n^+(\mathbb{C})$ est un ouvert, i.e, $\mathcal{M}_n(\mathbb{C})\backslash\mathcal{D}_n^+(\mathbb{C})$ est fermé. Supposons par l'absurde qu'on dispose de $(A_k)\in\mathcal{D}_n^+(\mathbb{C})^\mathbb{N}$ tel que $A_k\to A$ avec $A\in\mathcal{D}_n^+(\mathbb{C})$, i.e, $\chi_A$ à racines simples. Pour tout $k$, on note $\chi_{A_k} = \prod_{i=1}^n (X-\lambda_{i,k})$, par continuité de $\chi$, $\chi_{A_k}\to \chi_A$. Nécessairement, les $(\lambda_{i,k})$ vont tendre vers les $n$-racines distinctes de $\chi_A$, c'est qu'APCR, les $(\lambda_{i,k})$ vont être distincts deux à deux, et donc qu'à partir de ce même rang, les $A_k$ est diagonalisable. Absurde.  


	Montrons que $\inte \mathcal{D}_n(\mathbb{C}) = \mathcal{D}_n^+(\mathbb{C})$. On a bien $\mathcal{D}_n^+\subset \inte\mathcal{D}_n(\mathbb{C})$ puisque $\mathcal{D}_n^+(\mathbb{C})$ est un ouvert inclu dans $\mathcal{D}_n(\mathbb{C})$. 

	Pour l'autre inclusion, montrons que $\inte \mathcal{D}_n(\mathbb{C}) \subset \mathcal{D}_n^+(\mathbb{C})$, i.e, $\mathcal{M}_n(\mathbb{C})\backslash \mathcal{D}_n^+(\mathbb{C}) \subset \adh (\mathcal{M}_n(\mathbb{C})\backslash\mathcal{D}_n(\mathbb{C}))$, i.e $\mathcal{D}_n(\mathbb{C})\backslash\mathcal{D}_n^+(\mathbb{C})\subset \adh (\mathcal{M}_n(\mathbb{C})\backslash\mathcal{D}_n(\mathbb{C}))$. 

	Soit $A\in\mathcal{D}_n(\mathbb{C})\backslash\mathcal{D}_n^+(\mathbb{C})$, alors 
	\begin{align*}
	A \sim \diag\{\underbrace{\lambda, \ldots, \lambda}_{p}, \lambda_1, \ldots,\, \lambda_r\}
	\end{align*}

	avec $p\geq 2$. Posons pour tout $k\in\mathbb{N}^*$, 
	\begin{align*}
		A_k = A + E_{1, 2}\frac{1}{k}
	\end{align*}

	Alors pour tout $k\in\mathbb{N}^*$, $A_k$ n'est pas diagonalisable car $\dim E_{\lambda}(A_k) = p-1 < p$ alors que $p$ est la multiplicité de $\lambda$ dans $\chi_{A_k}$. De plus, $A_k\to A$. Ainsi, on a prouvé que $\mathcal{D}_n(\mathbb{C})\backslash\mathcal{D}_n^+(\mathbb{C}) \subset \adh(\mathcal{M}_n(\mathbb{C})\backslash\mathcal{D}_n(\mathbb{C}))$.

        Quelques précisions ont été faîtes par M. Monier que je n'ai pas précisé ici, les raisonnements peuvent être incomplets par moment.
	\end{document}

