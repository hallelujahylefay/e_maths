\documentclass{article}
\usepackage[utf8]{inputenc}
\usepackage{amsmath}
\usepackage{amsmath}
\usepackage{amssymb}
\usepackage{french}
\usepackage{stmaryrd}
\usepackage{geometry}
\geometry{hmargin=2.5cm,vmargin=1.5cm}
\newcommand*{\QED}{\hfill\ensuremath{\blacksquare}}%
\DeclareMathOperator{\im}{im}
\DeclareMathOperator{\diag}{diag}
\DeclareMathOperator{\tr}{Tr}
\begin{document}
\title{FICHE 08-01 : $\tr(A^p)$}
\author{Yvann Le Fay}
\date{Novembre 2019}
\maketitle
\section*{Enoncé}
Soit $A\in \mathcal{M}_n(\mathbb{K})$, montrer que 

\begin{align*}
	\forall p \in \mathbb{N}^*, \tr{A^p} = 0 \Longleftrightarrow A \textup{ est nilpotente}.
\end{align*}
\section*{Solution}
Si $A$ est nilpotente alors $A$ est semblable à une matrice triangulaire supérieure et le résultat est vrai.

Supposons que pour tout $p\in \mathbb{N}^*, \tr{A^p} = 0$ et que $A$ n'est pas nilpotente. Notons $\lambda_1, \ldots, \,\lambda_r$ ses valeurs propres distinctes non nulles et $\alpha_1, \ldots, \,\alpha_r$ les multiplicités associées. La trigonalisation de $A$ montre que $A^p$ admet $\lambda_1^p, \ldots, \,\lambda_r^p$ comme valeurs propres et alors $A^p$ est semblable à 
\begin{align*}
	\begin{pmatrix}
		\diag\{\lambda_1^p, \ldots\} & B_p \\
		0 & L^p
	\end{pmatrix}
\end{align*}

avec $L$ une matrice triangulaire supérieure stricte, donc $\tr A^p = \sum_{i=1}^r \lambda_i^p \alpha_i = 0$, ceci pour tout $p\in \mathbb{N}^*$. Ce système est un système de Vandermonde (et donc en vérité, pour $p\in \llbracket 1; n+1\rrbracket$ suffit), on en déduit que $\alpha_1 = \ldots  = \alpha_p = 0$ et donc $A$ n'a que des valeurs propres nulles, i.e $A$ est nilpotente. 

\QED
\end{document}
