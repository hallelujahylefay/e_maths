\documentclass{article}
\usepackage[utf8]{inputenc}
\usepackage{amsmath}
\usepackage{amsmath}
\usepackage{amssymb}
\usepackage{french}
\usepackage{stmaryrd}
\usepackage{geometry}
\geometry{hmargin=2.5cm,vmargin=1.5cm}
\newcommand*{\QED}{\hfill\ensuremath{\blacksquare}}%
\begin{document}
\title{FICHE 01-02 : Inégalité sur une relation d'équivalence : ALG1-02 1.1}
\author{Yvann Le Fay}
\date{Juin 2019}
\maketitle

\section*{Enoncé}
Soit $A$ un ensemble de cardinal $n$, $\mathcal{R}$ une relation d'équivalence sur $A$ avec $k$ classes, $m$ le cardinal du graphe de $\mathcal{R}$. Montrer que $n^2\leq km$
\section*{Solution}
On sait que les $k$ classes d'équivalence forment une partition de $A$, notons ces classes, $(A_i)_{1\leq i\leq k}$
\begin{align*}
n = \sum_{j=1}^k |A_i|
\end{align*}

Aussi, 
\begin{align*}
m &= |\{(x,y)\in A^2 : x \mathcal{R} y\}|\\
&=\sum_{x\in A} |\bar{x}|\\
&=\sum_{j=1}^k \sum_{x\in A_i}{\bar{x}}\\
&=\sum_{j=1}^k |A_i|^2 
\end{align*}

Puis par Cauchy-Schwarz, 
\begin{align*}
n^2 =\bigg( \sum_{j=1}^k |A_i|\bigg)^2 \leq k \sum_{j=1}^k |A_i|^2 = km
\end{align*}
\QED
\end{document}