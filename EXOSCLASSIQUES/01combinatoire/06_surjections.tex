\documentclass{article}
\usepackage[utf8]{inputenc}
\usepackage{amsmath}
\usepackage{amsmath}
\usepackage{amssymb}
\usepackage{french}
\usepackage{stmaryrd}
\usepackage{geometry}
\geometry{hmargin=2.5cm,vmargin=1.5cm}
\newcommand*{\QED}{\hfill\ensuremath{\blacksquare}}%
\begin{document}
\title{FICHE 01-06 : Dénombrement du nombre de surjections : ALG? K-08-D}
\author{Yvann Le Fay}
\date{Juin 2019}
\maketitle
\section*{Enoncé}
Dénombrer le nombre de surjections de $\llbracket 1;n\rrbracket$ dans $\llbracket 1;m\rrbracket$.
\section*{Solution (par la formule du Crible)}
Posons pour tout $k\in\llbracket 1;m\rrbracket$,
\begin{align*}
A_k = \{f : \llbracket 1;n\rrbracket \to \llbracket 1;m\rrbracket : \forall i\in\llbracket 1;n\rrbracket, \, f(i) \neq k\}
\end{align*}

Alors il est clair que le nombre de surjections est égal à 
\begin{align*}
\bigg|\overline{\bigcup_{k=1}^m A_k}\bigg| & = m^n - \sum_{k=1}^m (-1)^{k-1}\sum_{1\leq i_1<\ldots<i_k\leq n}\bigg|\bigcap_{j=1}^kA_{i_j}\bigg|\\
&=m^n +\sum_{k=1}^m (-1)^k \begin{pmatrix}
n\\k
\end{pmatrix} (m-k)^n\\
&=\sum_{k=0}^m (-1)^k \begin{pmatrix}
n\\k
\end{pmatrix}
(m-k)^n 
\end{align*}


\QED

\end{document}
