\documentclass{article}
\usepackage[utf8]{inputenc}
\usepackage{amsmath}
\usepackage{amsmath}
\usepackage{amssymb}
\usepackage{french}
\usepackage{stmaryrd}
\usepackage{geometry}
\geometry{hmargin=2.5cm,vmargin=1.5cm}
\newcommand*{\QED}{\hfill\ensuremath{\blacksquare}}%
\begin{document}
\title{FICHE 01-07 : Dénombrement du nombre d'involutions : ALG?}
\author{Yvann Le Fay}
\date{Juin 2019}
\maketitle
\section*{Enoncé}
Soit $\mathfrak{I}_n = \{\sigma\in \mathfrak{S}_n : \sigma^2 = \textup{Id}_n\}$, calculer $|\mathfrak{I}_n|$. Généraliser à $\mathfrak{I}_n^m = \{\sigma \in\mathfrak{S}_n : \sigma^m = \textup{Id}_n\}$. Calculer le nombre d'éléments d'ordre $p$ dans $\mathfrak{S}_{2p}$ avec $p$ premier.
\section*{Solution}
Soit $\sigma\in\mathfrak{I}_n$, notons $k=|\textup{supp }\sigma|$, $k$ est clairement pair, de plus, on a nécessairement
\begin{align*}
\sigma = \prod_{i \in \textup{supp }\sigma}(i,\sigma(i))
\end{align*}

avec les supports des transpositions disjoints entre-eux. Pour construire $\sigma$, il suffit donc de choisir les $k/2$-couples complètement disjoints, sans oublier que les produits vont permuter
\begin{align*}
\frac{1}{(k/2)!}\prod_{j=0}^{k/2-1}\begin{pmatrix}
n-2j\\2
\end{pmatrix}
\end{align*}

On trouve donc au final,
\begin{align*}
|\mathfrak{I}_n| = \sum_{k=2, \, k \in 2\mathbb{N}}\frac{1}{(k/2)!}\prod_{j=0}^{k/2-1}\begin{pmatrix}n-2j\\2\end{pmatrix}
\end{align*}


De même, on trouve facilement qu'un élément $\sigma$ de $\mathfrak{I}_n^m$ s'écrit
\begin{align*}
\sigma = \prod_{i\in \textup{supp }\sigma}(i,\sigma(i), \sigma^2(i),\ldots,\sigma^{m-1}(i))
\end{align*}

puis
\begin{align*}
|\mathfrak{I}_n^m|= \sum_{k=0,\, k\in m\mathbb{N}}\frac{1}{(k/m)!}\prod_{j=0}^{k/m-1}\begin{pmatrix}n-mj\\m\end{pmatrix}
\end{align*}

Non ! La généralisation précédente est totalement fausse, et pour cause, on peut essayer pour $m = n$, on devrait trouver $|\mathfrak{I}_n^n| = n!$, mais on trouve $2$ à la place. La première expression qu'on donne pour les involutions ne se généralise pas aussi facilement, il semble qu'aucune formule explicite pour $|\mathfrak{I}_n^m|$ n'ait été trouvée. 

Soit $\sigma\in \mathfrak{S}_{2p}$, un élément d'ordre $p$, alors on sait que $\sigma$ se décompose de manière unique à l'ordre prêt, comme un produit de cycles, dont l'ordre est le produit des plus petits communs multiples. Cela ne laisse que deux possibilités, soit il est de taille $p$, soit il est le produit de deux cycles de taille $p$. Pour le premier choix, on a $\begin{pmatrix}2p\\p\end{pmatrix}(p-1)!$ choix, pour le second, on a $\begin{pmatrix} 2p\\p\end{pmatrix}(p-1)!^2/2$, où la divison par 2 provient du fait que le produit commute. Ainsi, 

\begin{align*}
	|\{\sigma\in \mathfrak{S}_{2p} : o(\sigma)=p\}| = \begin{pmatrix} 2p\\p\end{pmatrix}(p-1)!(1+(p-1)!/2)
\end{align*}

Plus généralement, le nombre d'éléments d'ordre $p$ dans $\mathfrak{S}_{lp}$ est 
\begin{align*}
	\sum_{k=1}^l \frac{(p-1)!^k}{k!}\prod_{j=0}^{k-1}\begin{pmatrix} p(l-j)\\p\end{pmatrix}
\end{align*}

\QED

\end{document}
