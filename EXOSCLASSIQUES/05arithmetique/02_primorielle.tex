\documentclass{article}
\usepackage[utf8]{inputenc}
\usepackage{amsmath}
\usepackage{amsmath}
\usepackage{amssymb}
\usepackage{french}
\usepackage{stmaryrd}
\usepackage{geometry}
\geometry{hmargin=2.5cm,vmargin=1.5cm}
\newcommand*{\QED}{\hfill\ensuremath{\blacksquare}}%
\begin{document}
\title{FICHE 05-02 : Majoration de la primorielle ALG? MET 2. SE 1. K-12-2-8}
\author{Yvann Le Fay}
\date{Juin 2019}
\maketitle

\section*{Enoncé}
Soit $k\geq 1$, montrer que $\begin{pmatrix}
2k+1\\k
\end{pmatrix}< 4^k$ et en déduire que $P_n = \prod_{p\leq n}p< 4^n$.
\section*{Solution}
On a directement,
\begin{align*}
2\begin{pmatrix}
2k+1\\k
\end{pmatrix}=\begin{pmatrix}
2k+1\\k
\end{pmatrix}+\begin{pmatrix}
2k+1\\
k
\end{pmatrix}<\sum_{k=0}^{2k+1} \begin{pmatrix}
2k+1\\
j
\end{pmatrix}=2\times 4^k
\end{align*}

Montrons par récurrence que $P_n< 4^n$. La propriété est initialisée en $n=2$, supposons celle-ci vraie pour tout $j\leq n-1$. On vérifie par Gauss que pour $k+2\leq p\leq 2k+1$, $p\mid \begin{pmatrix}2k+1\\k\end{pmatrix}$, donc $p\leq 4^k$. . Si $n$ est pair alors $P_{n}=P_{n-1}<4^{n-1}<4^n$. Si $n=2k+1$ alors 
\begin{align*}
P_{n}=\prod_{1\leq p\leq k+1}p \prod_{k+2\leq p \leq 2k+1}p< 4^{k+1}\times 4^{k}=4^n
\end{align*}

Ce qui conclut la récurrence.
\QED
\end{document}