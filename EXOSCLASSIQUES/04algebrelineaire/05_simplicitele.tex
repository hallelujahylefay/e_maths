\documentclass{article}
\usepackage[utf8]{inputenc}
\usepackage{amsmath}
\usepackage{amsmath}
\usepackage{amssymb}
\usepackage{french}
\usepackage{stmaryrd}
\usepackage{geometry}
\geometry{hmargin=2.5cm,vmargin=1.5cm}
\newcommand*{\QED}{\hfill\ensuremath{\blacksquare}}%
\DeclareMathOperator{\im}{im}
\DeclareMathOperator{\rg}{rg}
\begin{document}
\title{FICHE 04-05 : Simplicité de $\mathcal{L}(E)$ : Troesch 24.15, misc}
\author{Yvann Le Fay}
\date{Juillet 2019}
\maketitle
\section*{Enoncé}
Montrer que les seuls idéaux bilatères de $\mathcal{L}(E)$ sont $\{0\}$ et $\mathcal{L}(E)$.
\section*{Solution}
Premièrement, on vérifie aisément que les deux propositions sont bien des idéaux bilatères de $\mathcal{L}(E)$. 

Soit $\mathcal{I}$ un idéal bilatère de $\mathcal{L}(E)$. Supposons que $\mathcal{I}\neq \{0\}$, alors il existe $u\in\mathcal{I}$ telle que $u\neq 0$. Introduisons une base de $E$ et la base canonique de $\mathcal{L}(E)$, $(e^*_{i,j})_{(1\leq i,j\leq n)}$ définie par 
\begin{align*}
\forall i,j\in\llbracket 1;n\rrbracket,\, e^*_{i,j}(e_k) = \delta_{i,k}e_j
\end{align*}

Pour tout $i,j,k,l, r \in \llbracket 1;n\rrbracket$, $e^*_{i,j}e^*_{k,l}(e_r) = e^*_{i,j}(\delta_{r,k}e_l) = \delta_{r,k} \delta_{l,i}e_j = \delta_{l,i}e^*_{k,j}(e_r)$

On décompose $u$ sur cette base, 
\begin{align*}
u = \sum_{i,j}\lambda_{i,j} e^*_{i,j}
\end{align*}

Or $u$ étant non nulle, il existe $i_0, j_0 \in\llbracket 1;n\rrbracket$ tels que $\lambda_{i_0,j_0}\neq 0$. Ainsi,

\begin{align*}
e^*_{j_0,j}ue^*_{i,i_0} = \lambda_{i_0,j_0}e_{i,j}
\end{align*}

On en déduit que $\mathcal{L}(E)\subset \mathcal{I}$ puis $I=\mathcal{L}(E)$.
\QED
\end{document}