% vim:ft=tex:
%
\documentclass{article}
\usepackage[utf8]{inputenc}
\usepackage{amsmath}
\usepackage{amssymb}
\usepackage{french}
\usepackage{stmaryrd}
\usepackage{geometry}
\geometry{hmargin=2.5cm,vmargin=1.5cm}
\newcommand*{\QED}{\hfill\ensuremath{\blacksquare}}%
\DeclareMathOperator{\tr}{Tr}
\begin{document}
\title{FICHE 04-07 : Tout hyperplan de $\mathcal{M}_n(K)$ intersecte $\mathcal{G}\mathcal{L}_n(K)$: MET ? }
\author{Yvann Le Fay}
\date{Août 2019}
\maketitle 
\section*{Enoncé}
Soit $n\geq 2$, montrer que tout hyperplan de $\mathcal{M}_n(K)$ intersecte $\mathcal{G}\mathcal{L}_n(K)$.
\section*{Solution}
Soit $H$ un hyperplan de $\mathcal{M}_n(K)$. L'application 
\begin{align*}
	\varphi : \left\{
		\begin{array}{@{}l@{\thinspace}l}
			\mathcal{M}_n(K)\to \mathcal{M}_n(K)^*\\
			A \mapsto (X\mapsto \tr{AX})
		\end{array}
	\right.
\end{align*}

est un isomorphisme, en effet, les dimensions des deux espaces étant égales, il suffit de prouver l'injectivité de $\varphi$.

Soit $A\in \ker \varphi$, alors pour tout $i,\,j\in \llbracket 1;n\rrbracket$, $\tr{AE_{i,j}} = a_{i,j} = 0$, donc $A=0$. 

Ce résultat prouvé, on peut maintenant affirmer qu'il existe $A\in \mathcal{M}_n(K)$ tel que 
\begin{align*}
	H = \ker \{X \mapsto \tr{AX}\}
\end{align*}

Notons $r$ le rang de $A$, alors il existe $P,\, Q\in\mathcal{G}\mathcal{L}_n(K)$ telles que $A = PJ_rQ$. 

Aussi, $\tr{AX} = \tr{PJ_rQX}=\tr{J_rQXP}$. 

Il suffit donc de trouver $G\in \mathcal{G}\mathcal{L}_n(K)$ telle que $\tr{J_rG} = 0$, car alors 
\begin{align*}
	X = Q^{-1}GP^{-1}\in H\cap \mathcal{G}\mathcal{L}_n(K)
\end{align*}
conviendra. 

Posons, 
\begin{align*}
	G = \begin{pmatrix}
		0 & 0 & \ldots & \ldots & 0 & 1\\
		1 & 0 & \ddots & & & 0\\
		0 & 1 & 0 & & & \vdots\\
		\vdots & &\ddots &\ddots & &\vdots \\
		\vdots & &&\ddots & \ddots & \vdots \\
		0 & \hdots & \hdots & \hdots & 1 & 0 
	\end{pmatrix}
\end{align*}

Celle-ci convient. 

\QED
\end{document}
