\documentclass{article}
\usepackage[utf8]{inputenc}
\usepackage{amsmath}
\usepackage{amsmath}
\usepackage{amssymb}
\usepackage{french}
\usepackage{stmaryrd}
\usepackage{geometry}
\geometry{hmargin=2.5cm,vmargin=1.5cm}
\newcommand*{\QED}{\hfill\ensuremath{\blacksquare}}%
\DeclareMathOperator{\im}{im}
\begin{document}
\title{FICHE 04-03 : Racine carrée de $D$ : ALG1-01 6.34}
\author{Yvann Le Fay}
\date{Juillet 2019}
\maketitle
\section*{Enoncé}
Montrer qu'il n'existe pas d'opérateur $T\in \mathcal{L}(E)$ tel que $T^2 = D$, où $D$ est l'opérateur de dérivation.
\section*{Solution d'Algèbre 01}
Considérons la restriction, $t$, de $T$ à $\ker D^2 = \mathbb{C}_1[X]$. Alors $t^4=0$, ainsi l'indice de nilpotence d'après la fiche 01 de ce chapitre, de $t$ est majorée par $2$, la dimension de l'espace de restriction. Ainsi $t^2 = d = 0$, ce qui est absurde.
\section*{Deuxième solution}
Soit $u\in\mathcal{L}(E)$, posons pour tout $k\in\mathbb{N}$, $N_k = \ker u^k$. On obtient aisément que
\begin{align*}
N_0\subset N_1\subset \ldots
\end{align*}

De plus, s'il existe $k_0\in\mathbb{N}$ tel que $N_{k_0}=N_{k_0+1}$, alors la suite $(N_k)$ est stationnaire à partir de ce rang (à démontrer). Maintenant que cela est dit, appliquons-le à $T$, on obtient
\begin{align*}
\{0\}\subset \ker T\subset \mathbb{C}_0[X]\subset \ker T^3 \subset \ldots
\end{align*}

Cela laisse deux possibilités, $\ker T = \{0\}$ et $\ker T = \mathbb{C}_0[X]$, dans les deux cas, on obtient que la suite des $(\mathbb{C}_k[X])$ est stationnaire, absurde.
\QED
\end{document}