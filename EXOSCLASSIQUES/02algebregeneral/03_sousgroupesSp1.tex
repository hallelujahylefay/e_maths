\documentclass{article}
\usepackage[utf8]{inputenc}
\usepackage{amsmath}
\usepackage{amssymb}
\usepackage{french}
\usepackage{stmaryrd}
\usepackage{geometry}
\geometry{hmargin=2.5cm,vmargin=1.5cm}
\newcommand*{\QED}{\hfill\ensuremath{\blacksquare}}%
\begin{document}
\title{FICHE 02-03 : Sous-groupe de $\mathfrak{S}_p$ et réciproque du théorème de Lagrange : MET Algebre E7}
\author{Yvann Le Fay}
\date{Juin 2019}
\maketitle
\section*{Enoncé}
Soit $H$ un sous-groupe de $\mathfrak{S}_p$ tel que $[\mathfrak{S}_p\, : \, H] = \frac{|\mathfrak{S}_p|}{|H|}\leq p-1$. Montrer que $[\mathfrak{S}_p\, : \, H] \in \{1,2\}$. On montrera d'abord que les $p$-cycles sont dans $H$.
\section*{Solution}
Soit $\gamma\in\mathfrak{S}_p$, un $p$-cycle, introduisons $\gamma^i H$ pour $0\leq i \leq p-1$. Il existe alors $i,\,j\in \llbracket 0;p-1\rrbracket,\, i\neq j$ tels que $\gamma^i H \cap \gamma^j H\neq \varnothing$ car sinon
\begin{align*}
|\mathfrak{S}_p|\geq \sum_{k=0}^{p-1}|\gamma^k H| = p|H|
\end{align*}

Ainsi, $e\neq \gamma^{j-i}\in H$ puis $\langle \gamma\rangle \subset H$. Or les $3$-cycles peuvent s'écrire avec des $p$-cycles, en effet, 
\begin{align*}
(i \, j \, k) = (i\, k\, j \, a_{p-3}\, \ldots a_1)(j\, i\, k\, a_1,\ldots\, a_{p-3}).
\end{align*}

De plus les $3$-cycles génèrent $\mathfrak{A}_p$, ainsi $p!/2\leq |H|$, d'où $[\mathfrak{S}_p\, :\, H] \in \{1,2\}$. 

Remarquons que cela donne un contre-exemple à la réciproque du théorème de Lagrange, pour $p=5$, il n'existe pas de sous-groupe d'ordre $40$ ($120/40=3\leq 5$). 
\QED

\end{document}
