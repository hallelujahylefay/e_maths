\documentclass{article}
\usepackage[utf8]{inputenc}
\usepackage{amsmath}
\usepackage{amsmath}
\usepackage{amssymb}
\usepackage{french}
\usepackage{stmaryrd}
\usepackage{geometry}
\geometry{hmargin=2.5cm,vmargin=1.5cm}
\newcommand*{\QED}{\hfill\ensuremath{\blacksquare}}%
\begin{document}
\title{FICHE 02-02 : Nombre de dérangements : ALG? K-09-5-2}
\author{Yvann Le Fay}
\date{Juin 2019}
\maketitle
\section*{Enoncé}
Soit $(A,+,\times)$ un anneau unifère tel que,
\begin{align*}
\forall a,b\in A, \quad ab=\pm ba
\end{align*}

Montrer que $A$ est commutatif.
\section*{Solution}
Notons $Z(A)$ le centre de $A$ est $Z^*(A)$ l'anticentre de $A$. Montrons que $A=Z(A)\cup Z^*(A)$. Supposons par l'absurde qu'il existe $a\in A\backslash Z(A)\cup Z^*(A)$, c'est-à-dire qu'il existe $x,y\in A$ tels que 
\begin{align*}
\left\{
     \begin{array}{@{}l@{\thinspace}l}
     ax=xa, \quad ax\neq -xa\\
     ay=-ya, \quad ay\neq ya
     \end{array}
   \right.   
\end{align*}

En sommant, $a(x+y)=(x-y)a=\pm a(x-y)$, on trouve alors dans le cas $+$, $ay=-ay=ya$, dans le cas $-$, $ax=-ax$, absurde dans les deux cas.

On vérifie facilement que $Z(A)$ et $Z^*(A)$ sont des groupes, un résultat classique nous dit que l'union de deux groupes est un groupe si et seulement si l'un est inclus dans l'autre. Si $Z^*(A)\subset Z(A)$ alors c'est gagné, si $Z(A)\subset Z^*(A)$ alors $1\times 1 = -1\times 1$, d'où,
\begin{align*}
\forall a,b\quad ab=-ba=ba
\end{align*}
\QED
\end{document}