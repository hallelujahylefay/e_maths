\documentclass{article}
\usepackage[utf8]{inputenc}
\usepackage{amsmath}
\usepackage{amsmath}
\usepackage{amssymb}
\usepackage{french}
\usepackage{stmaryrd}
\usepackage{geometry}
\geometry{hmargin=2.5cm,vmargin=1.5cm}
\newcommand*{\QED}{\hfill\ensuremath{\blacksquare}}%
\DeclareMathOperator{\im}{im}
\DeclareMathOperator{\stab}{Stab}
\DeclareMathOperator{\orb}{Orb}
\begin{document}
\title{FICHE 02-11 : Lemme de Cauchy : ALG1-01 2.10}
\author{Yvann Le Fay}
\date{Juillet 2019}
\maketitle
\section*{Enoncé}
\begin{enumerate}
	\item Soit $G$ un groupe fini de cardinal $p^m$ avec $m\in\mathbb{N}^*$ et $p$ premier qui opère sur un ensemble fini non vide $E$, on note 
		\begin{align*}
			E^G = \{x\in E : \forall g \in G,\, gx = x\}
		\end{align*}
		Montrer que $|E^G| \equiv |E|\, [p]$
	\item Soit $H$ un groupe fini d'ordre $n$ et $p$ un diviseur premier de $n$. Montrer que $H$ contient un élément d'ordre $p$. On introduira une opération de $\mathbb{Z}/p\mathbb{Z}$ sur l'ensemble $E$ des $(x_1,\ldots x_p)\in H^p$ telle que $\prod_{i=1}^p x_i = e$.

\end{enumerate}
\section*{Solution}
$E^G$ n'est rien d'autre que l'ensemble des $x\in E$ tels que $\orb(x) = \{x\}$. De plus, on sait que les orbites forment une partition de $E$ et on a $\forall x \in G$, $|G|=|\stab(x)||\orb(x)|$, on en déduit l'équation aux classes suivante
\begin{align*}
	|E|=\sum_{i\in I} \frac{|G|}{|\stab{x_i}|} = |E^G|+\sum_{j=1}^n |w_j|
\end{align*}

où les $|w_j|$ pour $1\leq j \leq n$ sont les termes de la somme de gauche tels que $|\stab{x_i}|<|G|$. On en déduit par le théorème de Lagrange que $|\stab{x_i}| \in \{1,\ldots p^{m-1}\}$ puis que les $|w_j|$ sont des puissances de $p$, d'où le résultat.

Introduisons $(x_1,\ldots x_p)\in H^p$ tel que $x_1\ldots x_p = 1$ alors $x_2\ldots x_p x_1 = 1$, notons $c$ le cycle $(1, 2,\ldots p)$. On remarque que $K = \langle c\rangle$ est isomorphe à $\mathbb{Z}/p\mathbb{Z}$ et que pour tout $c'\in K$, $\prod_{i=1}^p x_{c'(i)} = 1$. Notons $E$ l'ensemble des $p$-plets de produit égal à $1$ et appliquons le résultat de la question précédente à l'opération 
\begin{align*}
 \left\{
     \begin{array}{@{}l@{\thinspace}l}
     K &\to E\\
     c'&\mapsto c(x_1,\ldots x_p) = (x_{c(1)},\ldots x_{c(n)})
     \end{array}
   \right.   
\end{align*}

On obtient donc que 
\begin{align*}
	|E| \equiv |E^K| \, [p]
\end{align*}

Or $|E| = n^{p-1}$ par un simple argument combinatoire (au choix $p-1$ éléments de $E$ puis le dernier est l'inverse). Or $p\mid n$ donc $|E^{K}| \equiv 0\, [p]$, or $E^K$ est non vide puisque $(e,\ldots e)$ en est un élément donc $|E^K|\geq p$. De plus, $E^K  =\{(x_1,\ldots x_p) \in E : \forall c'\in \langle c \rangle, (x_{c'(1)},\ldots x_{c'(p)}) = (x_1,\ldots x_p)\} = \{(x,\ldots x)\in H^p  : x^p = 1\}$. De plus, s'il existait $x\in H$ tel que $x^p = 1$ et $x$ n'est pas d'ordre $p$ alors l'ordre de $x$ diviserait $p$ sans être égal à $p$, d'où $x = e$. Donc mis à part $e$, $E^K$ contient l'ensemble des éléments d'ordre $p$. Ainsi on a prouvé qu'il y avait un nombre $kp-1$ éléments d'ordre $p$.
\QED
\end{document}
