\documentclass{article}
\usepackage[utf8]{inputenc}
\usepackage{amsmath}
\usepackage{amsmath}
\usepackage{amssymb}
\usepackage{french}
\usepackage{stmaryrd}
\usepackage{geometry}
\geometry{hmargin=2.5cm,vmargin=1.5cm}
\newcommand*{\QED}{\hfill\ensuremath{\blacksquare}}%
\DeclareMathOperator{\im}{im}
\begin{document}
\title{FICHE 02-08 : Les $p$-groupes: MET-1 1.2.10, ALG1-02 2.11}
\author{Yvann Le Fay}
\date{Juillet 2019}
\maketitle
\section*{Enoncé}
Soit $G$ un groupe d'ordre $p^m$, avec $p$ premier, montrer que $|Z(G)|=p^k$ avec $1\leq k \leq m$.

Soit $G$ un groupe d'ordre $p^2$, démontrer que $G$ est abélien.
\section*{Solution}
D'après l'équation aux classes, il existe une famille finie de sous groupes stricts de $G$, $(H_i)_{i\in I}$ tels que
\begin{align*}
p^m = |Z(G)| + \sum_{i\in I}\frac{p^m}{|H_i|}
\end{align*}
D'après le théorème de Lagrange, pour tout $i\in I$, $\frac{p^m}{|H_i|}\in \{1,\ldots,p^m\}$, le cas $1$ est à exclure car $H_i$ est un sous groupe strict, par l'équation aux classes, on en déduit que $p\mid |Z(G)|$, d'où le résultat. Dans le cas où $m=2$, supposons par l'absurde que $|Z(G)| = p$, il existe $x\in G\backslash Z(G)$, alors $N_x = \{g\in G : gx=xg\}$ est un sous-groupe strict de $G$. Aussi $Z(G)\subset N_x$ donc nécessairement $|N_x|=p$ puis $Z(G)=N_x$, ce qui est absurde.
\QED
\end{document}
