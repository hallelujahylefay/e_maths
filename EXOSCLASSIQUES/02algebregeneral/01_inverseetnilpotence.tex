\documentclass{article}
\usepackage[utf8]{inputenc}
\usepackage{amsmath}
\usepackage{amsmath}
\usepackage{amssymb}
\usepackage{french}
\usepackage{stmaryrd}
\usepackage{geometry}
\geometry{hmargin=2.5cm,vmargin=1.5cm}
\newcommand*{\QED}{\hfill\ensuremath{\blacksquare}}%
\begin{document}
\title{FICHE 02-01 : Calcul d'un inverse : ALG1-02 2.2 K-9-1-12}
\author{Yvann Le Fay}
\date{Juin 2019}
\maketitle

\section*{Enoncé}
Soit $G$, un groupe, $a,b\in A$, tels que $1-ab$ soit inversible, calculer $(1-ba)^{-1}$.
\section*{Solution}
L'idée est que si $a$ est un élément nilpotent, alors l'inverse de $1-ab$ s'écrit par Bernoulli,
\begin{align*}
\sum_{k=0}^\infty (ab)^k 
\end{align*}

On en déduit que $ba$ est nilpotent puis que l'inverse de $1-ba$ est
\begin{align*}
1+ba+(ba)^2+\ldots &= 1+b(1+ab+\ldots)a\\
&=1+b(1-ab)^{-1}a
\end{align*}

En général, on vérifie sans mal que $(1-ba)(1+b(1-ab)^{-1}a) = 1$ \QED
\end{document}