\documentclass{article}
\usepackage[utf8]{inputenc}
\usepackage{amsmath}
\usepackage{amsmath}
\usepackage{amssymb}
\usepackage{french}
\usepackage{stmaryrd}
\usepackage{geometry}
\geometry{hmargin=2.5cm,vmargin=1.5cm}
\newcommand*{\QED}{\hfill\ensuremath{\blacksquare}}%
\begin{document}
\title{FICHE 02-05 : Groupe abélien d'ordre $pq$ : ALG1-01-2.6}
\author{Yvann Le Fay}
\date{Juin 2019}
\maketitle
\section*{Enoncé}
Soit $G$ un groupe abélien d'ordre $pq$ avec $p$ et $q$ premiers distincts. Montrer que $G$ est cyclique.
\section*{Solution}
Par le Lemme de Cauchy, il existe un élément $x$ d'ordre $p$ et un élément $y$ d'ordre $q$, les deux sont premiers entre-eux et commutent, ainsi $xy$ est d'ordre $pq$.
On peut préciser la démonstration que $xy$ est d'ordre $pq$, en effet, $(xy)^p = y^p \neq e$ car $q\nmid p$, de même pour $q$, et $xy\neq e$ car sinon $p=q$, ainsi par le théorème de Lagrange, l'ordre de $xy$ ne peut qu'être $pq$.
\QED

\end{document}