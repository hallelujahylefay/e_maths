\documentclass{article}
\usepackage[utf8]{inputenc}
\usepackage{amsmath}
\usepackage{amsmath}
\usepackage{amssymb}
\usepackage{french}
\usepackage{stmaryrd}
\usepackage{geometry}
\geometry{hmargin=2.5cm,vmargin=1.5cm}
\newcommand*{\QED}{\hfill\ensuremath{\blacksquare}}%
\DeclareMathOperator{\im}{im}
\DeclareMathOperator{\diag}{diag}
\begin{document}
\title{FICHE 07-02 : Déterminant de Vandermonde lacunaire ALG2-01 1-11}
\author{Yvann Le Fay}
\date{Juillet 2019}
\maketitle
\section*{Enoncé}
Soit $0\leq k\leq n$, calculer
\begin{align*}
D_k = 
\begin{vmatrix}
1 & x_1 & \ldots & x_1^{k-1} & x_1^{k+1}& \ldots & x_1^n\\
\vdots & & & \vdots& \vdots & & \vdots\\
1 & x_n & \ldots & x_n^{k-1} & x_n^{k+1}& \ldots & x_n^n
\end{vmatrix}
\end{align*}

\section*{Solution}
Introduisons le déterminant de Vandermonde associé, en rajoutant la colonne manquante et la ligne manquante, on a
\begin{align*}
V = \begin{vmatrix}
1 & x_1 & \ldots & x_1^k & \ldots & x_1^n\\
\vdots & & \vdots& \vdots\\
1 & x_n & \ldots &x_n^k&  \ldots & x_n^n\\
1 & X &\ldots &X^k&\ldots &X^n
\end{vmatrix}
\end{align*} 

On a classiquement que, 
\begin{align*}
V &= \prod_{1\leq i<j\leq n}(x_j-x_i)\prod_{k=1}^n (X-x_k)\\
&=\prod_{1\leq i<j\leq n}(x_j-x_i)\bigg(X^n -\sigma_1 X^{n-1}+\ldots+(-1)^n \sigma_n\bigg)
\end{align*}

Avec $\sigma_k$, les fonctions élémentaires des $x_k$. 
De plus, $D_k$ est le mineur de la colonne $k+1$, ligne, $n+1$, ainsi, $(-1)^{n+k}D_k$ est le coefficient de degré $k$ de $V$, i.e, 
\begin{align*}
D_k = \sigma_{n-k} \prod_{1\leq i<j\leq n}(x_j-x_i)
\end{align*}
\QED
\end{document}
