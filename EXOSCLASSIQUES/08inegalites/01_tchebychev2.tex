\documentclass{article}
\usepackage[utf8]{inputenc}
\usepackage{amsmath}
\usepackage{amsmath}
\usepackage{amssymb}
\usepackage{french}
\usepackage{stmaryrd}
\usepackage{geometry}
\geometry{hmargin=2.5cm,vmargin=1.5cm}
\newcommand*{\QED}{\hfill\ensuremath{\blacksquare}}%
\begin{document}
\title{FICHE 08-01 : Une inégalité de Tchebychev : exercice 404 d'Alix}
\author{Yvann Le Fay}
\date{Juillet 2019}
\maketitle

\section*{Enoncé}
Soient $f$, $g$ deux fonctions continues de même monotonie et $a \leq b$, montrer que 
\begin{align*}
\int_{a}^b f\int_{a}^b g \leq (b-a)\int_{a}^b fg
\end{align*}
\section*{Solution}
L'idée est d'utiliser les sommes de Riemann, l'inégalité est en effet impliquée par 
\begin{align}
\bigg(\frac{b-a}{n}\bigg)^2\sum_{k=0}^{n-1} f_k \sum_{k=0}^{n-1} g_k \leq \frac{(b-a)^2}{n}\sum_{k=0}^{n-1} {f_k g_k}
\end{align}

où pour tout $k\in\llbracket 0;n\rrbracket$, $f_k = f(a+k\frac{b-a}{n})$, similairement pour $g_k$. Afin de la montrer, on remarque que l'hypothèse sur la monotonie implique que pour tout $k,j\in\llbracket 0;n-1\rrbracket$
\begin{align*}
(f_k-f_j)(g_k-g_j)\geq 0 
\end{align*}

Ainsi, 
\begin{align*}
&\frac{b-a}{n}\sum_{k,j}(f_k-f_j)(g_k-g_j) = \frac{b-a}{n}\bigg(2n\sum_k f_kg_k-2\sum_{k,j} f_kg_j\bigg) \geq 0\\
\Longleftrightarrow &(b-a)\sum_k f_kg_k \geq \frac{b-a}{n}\sum_{k,j}f_kg_j\\
\Longleftrightarrow & (1)
\end{align*}
\QED
\end{document}